Os smartphones fazem parte do cotidiano da sociedade, de modo que muitas tarefas podem ser realizadas por intermédio deles.
Comunicação, entretenimento, notícias, educação, saúde, serviços, entre outros.
As operações por meio de smartphones já representam mais de 51\% das transações bancárias no Brasil.
Empresas do setor financeiro estão investindo muito neste segmento.
Estima-se que só em 2020 foram aplicados mais de R\$ 25,7 bilhões~\cite{febraban-data}.
Segundo uma pesquisa realizada em 2022, existem 447 milhões de dispositivos digitais em uso no Brasil, sendo 242 milhões destes, smartphones.
Há mais de 1 smartphone por habitante em território nacional~\cite{it-usage-data}.
O Banco Central do Brasil estabeleceu em 2020 um novo meio de pagamento — o Pix.
Esse meio de pagamento nasceu com a ideia de utilizar o smartphone para enviar e receber dinheiro, dada a sua massiva utilização.
Apenas no mês março de 2022, mais de 784 milhões de transações foram realizadas por meio do Pix~\cite{pix-statistics}.

Dada a importância que os smartphones têm na sociedade, é importante garantir que as aplicações desenvolvidas tenham um padrão de qualidade.
Empresas de tecnologia estão preocupadas com o \emph{time to market}, termo que designa o tempo decorrido desde a concepção da ideia até a publicação de um produto.
Assim, diversas abordagens são utilizadas para entregar o produto no menor tempo possível, em muitos casos sacrificando a qualidade, mas que geram débitos técnicos dentro do projeto.
Em geral, nota-se que mudanças na regra de negócio e eventuais adições de funcionalidades podem ter uma complexidade maior quando a arquitetura do projeto é muito permissiva, ou não é bem definida.
A garantia de qualidade muitas vezes está concentrada apenas no processo de QA, e o código-fonte produzido não é levado em consideração.
Nem tudo pode ser prevenido, e \emph{bugs} geralmente são um reflexo disso~\cite{quality-standards-paper}.