Os \emph{smartphones} fazem parte do cotidiano da sociedade, de modo que muitas tarefas podem ser realizadas por intermédio deles.
Comunicação, entretenimento, notícias, educação, saúde, serviços, entre outros.
De acordo com uma pesquisa realizada pela FEBRABAN, as operações por meio de \emph{smartphones} já representam mais de 51\% das transações bancárias no Brasil.

Empresas do setor financeiro estão investindo muito neste segmento.
Estima-se que só em 2020 foram aplicados mais de R\$ 25,7 bilhões~\cite{febraban-data}.
Segundo uma pesquisa realizada em 2022 pela Fundação Getulio Vargas, existem 447 milhões de dispositivos digitais em uso no Brasil, sendo 242 milhões destes, \emph{smartphones}.
Há mais de 1 \emph{smartphone} por habitante em território nacional~\cite{it-usage-data}.

Tendo em vista a massiva utilização dos \emph{smartphones} no Brasil, em 2020, o Banco Central do Brasil estabeleceu o Pix, um novo meio de pagamento que utiliza esses dispositivos para enviar e receber dinheiro.
Apenas no mês março de 2022, mais de R\$ 787 milhões foram transacionados por meio do Pix~\cite{pix-statistics}.
Dada a importância que os \emph{smartphones} têm na sociedade, é importante garantir que as aplicações desenvolvidas tenham um padrão de qualidade.
Empresas de tecnologia estão preocupadas com o \emph{time to market}, termo que designa o tempo decorrido desde a concepção da ideia até a publicação de um produto.

Diversas abordagens são utilizadas para entregar o produto no menor tempo possível, em muitos casos sacrificando a qualidade, e causando débitos técnicos dentro do projeto.
Em geral, nota-se que mudanças na regra de negócio e eventuais adições de funcionalidades podem ter uma complexidade maior quando a arquitetura do projeto é muito permissiva, ou não é bem definida.
A garantia de qualidade muitas vezes está concentrada apenas nos testes realizados por analistas de qualidade de software, e o código-fonte produzido não é levado em consideração.
Nem tudo pode ser prevenido, e \emph{bugs} geralmente são um reflexo disso~\cite{quality-standards-paper}.