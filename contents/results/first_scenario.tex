No MVC, as operações relacionadas ao banco de dados remoto se encontram nos \emph{Models}, os quais são \emph{NoteModel} e \emph{UserModel}.
Toda a lógica para obter e adicionar dados está concentrada nestes dois componentes.
Eles fazem uso direto das bibliotecas externas que são necessárias para acessar o Firebase.
Seria necessário alterar esses dois componentes quase que em sua totalidade para se adequarem ao cenário proposto.
Com a alteração destes dois componentes, também seria necessário efetuar modificações nos \emph{Controllers} da aplicação que interagem diretamente com os \emph{Models}.

Na arquitetura limpa, as operações relacionadas ao banco de dados remoto se encontram na camada de agentes externos.
Existem duas classes abstratas que definem operações no banco de dados remotos: \emph{NotesRemoteDataBase} e \emph{UsersRemoteDataBase}.
Essas duas classes possuem suas respectivas implementações utilizando o Firebase.
Os repositórios que fazem uso do banco de dados remoto recebem uma instância das implementações por meio de injeção de dependências, pois estão dependendo de uma abstração, e não de uma implementação concreta.
Para se adequar ao cenário proposto, seria necessário criar implementações de \emph{NotesRemoteDataBase} e \emph{UsersRemoteDataBase} no novo banco de dados remoto, e realizar a injeção deles na aplicação.
Não é necessário alterar nenhum componente.