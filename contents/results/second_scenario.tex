No MVC, para adequar-se ao cenário proposto, é necessário criar dois métodos em \emph{NoteModel} para realizar a busca, tanto no banco de dados local como no remoto.
Os dois métodos farão uso direto das bibliotecas externas que são necessárias.
Um terceiro método para ordenar os resultados de forma cronológica também poderá ser criado, e utilizado em ambos os métodos de busca.
Além disso, é necessário criar novos métodos no \emph{Controller} responsável pela listagem de notas, de modo a executar os métodos de busca presentes em \emph{NoteModel}.
A \emph{View} que realiza a listagem de notas também sofrerá alterações.

Já na arquitetura limpa, é necessário criar um novo caso de uso que realize a validação dos dados de entrada do usuário, solicite dados ao repositório das notas e que ao recebê-los, realize a sua ordenação de forma cronológica.
A abstração e implementação do repositório de notas deverão ser modificadas, adicionando-se um método para realização da busca.

O mesmo deve acontecer com as abstrações e implementações dos bancos de dados, que por sua vez, farão uso direto das bibliotecas externas que são necessárias.
Assim como no MVC, é necessário criar novos métodos no \emph{Controller} responsável pela listagem de notas, de modo a consumir o recém criado caso de uso, bem como realizar alterações na \emph{View} correspondente.