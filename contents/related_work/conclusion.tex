O presente artigo justifica-se por trazer uma análise baseada em cenários, a partir de um projeto de exemplo, em cada uma das abordagens selecionadas.
São considerados aspectos importantes, como a facilidade de implementar novos recursos, o impacto de modificações futuras, possibilidade de modularizar a aplicação, curva de aprendizado de cada abordagem e a facilidade de escrever testes unitários.
Também, a análise realizada é válida tanto no desenvolvimento em plataformas específicas, como o Android e iOS, quanto em frameworks híbridos, como Flutter ou React Native.
De acordo com o conhecimento dos autores dos artigos citados, não foi realizado um comparativo para aplicativos de smartphones com este objetivo.