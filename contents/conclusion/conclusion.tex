No presente artigo, o padrão MVC e a arquitetura limpa são comparadas por meio do método de análise de arquiteturas de software baseado em cenários.
São utilizados três cenários: mudanças no banco de dados remoto, implementação da pesquisa de notas por título e conteúdo, e por fim, a implementação de um novo método de autenticação.
Em cada cenário, são observados os componentes que precisam ser criados ou alterados.

A partir da análise realizada, é possível observar que cada abordagem possui seus pontos positivos e negativos.
O MVC permite que as aplicações sejam desenvolvidas com mais facilidade e rapidez, pois não existem muitos limites arquiteturais a serem respeitados.
A lógica da aplicação se concentra nas \emph{Models}, e em alguns casos, também nos \emph{Controllers}.
Entretanto, num cenário em que é preciso realizar modificações profundas, como em uma mudança de banco de dados remoto, as \emph{Models} são reescritas quase que em sua totalidade, e os \emph{Controllers} precisam adaptar-se às mudanças que acontecem.
Isso ocorre pois os \emph{Models} fazem uso direto de agentes externos, como bibliotecas de terceiros, sem qualquer camada de abstração, o que os deixam mais suscetíveis a falhas inesperadas.
Pelo mesmo motivo, escrever testes unitários pode ser um desafio.
Caso a dependência externa não possua uma forma de simular o seu comportamento por meio de uma classe de \emph{Mock}, nem sempre será possível realizar um teste unitário.
Neste cenário, o teste de integração se torna a melhor opção para garantir a corretude do código, mas com a desvantagem de ser mais custoso.
Além disso, é possível observar que as \emph{Models} tendem a ficar cada vez maiores com a adição de novas funcionalidades, o que pode dificultar a legibilidade e manutenção do código, além de atribuir muitas responsabilidades à um único componente.

Por sua vez, a arquitetura limpa torna o processo de desenvolvimento mais demorado, já que abstrações são utilizadas em várias camadas, e há a necessidade de isolar serviços externos do domínio da aplicação.
Seguir por essa abordagem facilita a realização de mudanças mais complexas, conforme observado no cenário da mudança do banco de dados.
A lógica da aplicação se concentra nas entidades e nos casos de uso, que não são afetados quando alguma dependência externa sofre alterações.
Assim, é possível realizar testes unitários com facilidade, pois os componentes que lidam com as regras de negócio estão isolados, e não dependem de nada externo para funcionarem corretamente.
Entretanto, novas funcionalidades na aplicação exigem a criação de muitos componentes, bem como as interfaces que realizarão a ponte entre as camadas.
Pode ser um desafio para desenvolvedores que não estão familiarizados entender todos os conceitos por trás desta abordagem, já que existem muitos componentes com responsabilidades distintas.

De modo geral, o desenvolvimento utilizando a arquitetura limpa é mais demorado em relação ao MVC, além de necessitar que as regras de negócio sejam muito bem definidas antes do desenvolvimento ser iniciado.
Porém, uma vez desenvolvida, é mais simples realizar a manutenção e os testes unitários da aplicação, já que os componentes estão bastante desacoplados entre si.
Já no MVC, o inverso ocorre: é mais rápido desenvolver, mas a manutenção e testes unitários são mais custosos.

Do ponto de vista da experiência de desenvolvimento, o código produzido utilizando o MVC tende a ser extenso, com classes que possuem muitos atributos e métodos.
No caso da arquitetura limpa, há a necessidade de se criar muitas classes concretas e abstratas, e por esse motivo, muitas pastas e arquivos diferentes são criados.
Essa característica permite que um time trabalhe em partes diferentes da mesma aplicação sem que hajam tantos conflitos, pois arquivos diferentes são utlizados.
Ainda, há a possibilidade de se implementar \emph{Micro Frontends} com mais facilidade, isto é, a modularização da aplicação em pequenos pacotes, um para cada funcionalidade, com um único módulo compartilhado entre todos os pacotes.

Em projetos de pequeno porte, provas de conceito e MVPs (\emph{Minimum Viable Product}, ou produto mínimo viável, em inglês), onde se deseja um rápido desenvolvimento, não se planeja realizar grandes mudanças no futuro, e a regra de negócio não é muito complexa, o MVC se torna uma ótima opção.
Em grandes projetos, onde a aplicação necessita de estabilidade, possui regras de negócio complexas, esteiras de testes unitários, ou até mesmo uma modularização em pequenos pacotes, a arquitetura limpa se torna essencial.