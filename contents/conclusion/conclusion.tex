No presente artigo, o padrão MVC e a arquitetura limpa são comparadas por meio do método de análise de arquiteturas de software baseado em cenários.
São utilizados três cenários: mudanças no banco de dados remoto, implementação da pesquisa de notas por título e conteúdo, e por fim, a implementação de um novo método de autenticação.
Em cada cenário, são observados os componentes que precisam ser criados ou alterados.

Em projetos de pequeno porte, provas de conceito e MVPs (\emph{Minimum Viable Product}, ou produto mínimo viável, em inglês), onde se deseja um rápido desenvolvimento, não se planeja realizar grandes mudanças no futuro, e a regra de negócio não é muito complexa, o MVC se torna uma ótima opção.
Em grandes projetos, onde a aplicação necessita de estabilidade, possui regras de negócio complexas, esteiras de testes unitários, ou até mesmo uma modularização em pequenos pacotes, a arquitetura limpa mostrou-se mais adequada.