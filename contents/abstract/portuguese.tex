Ao criar aplicações para smartphones, os desenvolvedores tomam decisões importantes que impactam diretamente no crescimento de um projeto no longo prazo.
A arquitetura de software é um fator determinante para o êxito de um aplicativo.
É de suma importância que esta seja bem definida e possua características que permitam sua evolução e manutenção de maneira sustentável.
Este trabalho compara duas abordagens de desenvolvimento amplamente empregadas, analisando suas características.
O padrão MVC é comparado com a Arquitetura Limpa, utilizando o método de análise de arquiteturas de software baseado em cenários.
A principal contribuição deste artigo é fornecer uma análise das arquiteturas escolhidas em uma aplicação para smartphones.