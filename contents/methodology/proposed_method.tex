A análise se dará por meio do método baseado em cenários.
Um cenário é uma breve narrativa de usos esperados ou antecipados do sistema, tanto do ponto de vista do usuário quanto do desenvolvedor.
Ele fornece uma visão de como o sistema satisfaz atributos de qualidade em vários contextos de uso~\cite{scenario-based}.

Inicialmente, é realizada uma descrição da arquitetura candidata.
Após esta etapa, os cenários são desenvolvidos, de modo a ilustrar os tipos de atividades que o sistema deve suportar e as mudanças que são esperadas ao longo da vida de um software.
Então, para cada cenário, é avaliado se a arquitetura necessita de alterações para executar o cenário.
Em caso afirmativo, o cenário é classificado como direto. Caso contrário, é classificado como indireto.

Quando dois ou mais cenários indiretos necessitam de mudanças em algum componente, diz-se que eles interagem.
As interações mostram quais módulos estão envolvidos em quais tarefas.
Um alto nível de interação pode significar que um componente possui um alto acoplamento.

\subsubsection{Cenários propostos}

Serão analisados os seguintes cenários:

\begin{enumerate}
	\item Mudança no banco de dados remoto

	      É desejado que o banco de dados remoto seja substituído, de modo a utilizar a infraestrutura de outro serviço externo.

	\item Implementação da pesquisa de notas por título e conteúdo

	      É desejado que seja possível realizar a pesquisa por notas, tanto no banco de dados remoto quanto no local. Os resultados devem ser apresentados em ordem cronológica.

	\item Implementação de um novo método de autenticação

	      É desejado que seja possível autenticar-se utilizando uma conta do GitHub, além dos métodos já existentes.
\end{enumerate}